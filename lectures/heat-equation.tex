\documentclass[10pt]{beamer}
\usepackage{tabulary}
\usepackage{amsfonts}
\usepackage{color}
\usepackage{amssymb}
\usepackage{amsmath}
\usepackage{amsthm}
\usepackage{epstopdf}
\usepackage{natbib}
\usepackage{setspace}
\usepackage{booktabs}
\usepackage{tikz}
\usepackage{graphicx}
\usepackage{tabulary}
\usepackage{caption} 

\usecolortheme{whale}
\usepackage{algorithm}
\usepackage{algpseudocode}
%\usefonttheme{professionalfonts}
%\setbeamerfont{itemize/enumerate body}{family=\sffamily, size={\fontsize{10}{10}}}
%\setbeamerfont{itemize/enumerate subbody}{family=\sffamily, size={\fontsize{10}{10}}}
\setbeamerfont{frametitle}{size=\large}

%\mode<presentation>
%\usetheme{CambridgeUS}
\usetheme{metropolis}

\title{The Heat Equation: Theory and Implementation}
\author{Graduate Numerical Methods}
\date{\today}

\usecolortheme{default} % change this to change the color scheme
\usepackage{istgame}
%\setbeamercovered{transparent}
\setbeamertemplate{navigation symbols}{}
\usepackage{caption,subcaption}
\usepackage[flushleft]{threeparttable}
\setbeamertemplate{itemize items}[circle]
\def\sym#1{\ifmmode^{#1}\else\(^{#1}\)\fi}
% Define document properties
\title{\textsc{The Heat Equation: Theory and Implementation}}

\date{\today}
\author[Short Name (U ABC)]{%
  \texorpdfstring{%
    \begin{columns}
      \column{\linewidth}
      Swapnil Singh \\ Bank of Lithuania, KTU
    \end{columns}
 }
 {Anu Alexander, Nancy Luke, Kaivan Munshi, Swapnil Singh}
}


\newcommand*\widebar[1]{%
   \hbox{%
     \vbox{%
       \hrule height 0.5pt % The actual bar
       \kern0.5ex%         % Distance between bar and symbol
       \hbox{%
         \kern-0.1em%      % Shortening on the left side
         \ensuremath{#1}%
         \kern-0.1em%      % Shortening on the right side
       }%
     }%
   }%
} 

% bibtex fix
\bibliographystyle{plainnat}


% theorems etc
\theoremstyle{plain}% default
\newtheorem{thm}{Theorem}
\newtheorem{lem}{Lemma}
%\newtheorem{prop}{Proposition}
\newtheorem{cor}{Corollary}
\theoremstyle{definition}
\newtheorem{defn}{Definition}
\newtheorem{conj}{Conjecture}
\newtheorem{exmp}{Example}
\theoremstyle{remark}
\newtheorem*{rem}{Remark}
%\newtheorem*{note}{Note}
\newtheorem{case}{Case}

% References to theorems etc
\newcommand{\thmref}[1]{Theorem~\ref{#1}}
\newcommand{\lemref}[1]{Lemma~\ref{#1}}
\newcommand{\propref}[1]{Proposition~\ref{#1}}
\newcommand{\corref}[1]{Corollary~\ref{#1}}
\newcommand{\defref}[1]{Definition~\ref{#1}}


\newcommand{\propnumber}{} % initialize
\newtheorem*{prop}{Proposition \propnumber}
\newenvironment{propc}[1]
{\renewcommand{\propnumber}{#1}%
\begin{prop}
		\end{prop}}


\begin{document}



\begin{frame}
    \titlepage
\end{frame}

\begin{frame}{Heat Equation: Mathematical Formulation}
    \begin{itemize}
        \item The heat equation describes how temperature distribution changes over time:
        \[ \frac{\partial u}{\partial t} = \alpha \frac{\partial^2 u}{\partial x^2} \]
        \item Where:
        \begin{itemize}
            \item $u(x,t)$ is temperature at position $x$ and time $t$
            \item $\alpha$ is thermal diffusivity coefficient
        \end{itemize}
        \vspace{0.5cm}
        \item \textbf{Physical Intuition:}
        \begin{itemize}
            \item Rate of change of temperature ($\frac{\partial u}{\partial t}$) is proportional to the curvature ($\frac{\partial^2 u}{\partial x^2}$)
            \item Heat flows from hot to cold regions
            \item System tends toward thermal equilibrium
        \end{itemize}
    \end{itemize}
\end{frame}

\begin{frame}{Finite Difference Approximation}
    \begin{itemize}
        \item We discretize space and time:
        \begin{itemize}
            \item Space: $x_i = i\Delta x$, where $i = 0,1,\ldots,N_x$
            \item Time: $t_n = n\Delta t$, where $n = 0,1,\ldots,N_t$
        \end{itemize}
        \vspace{0.3cm}
        \item Approximate derivatives:
        \begin{align*}
            \frac{\partial u}{\partial t} &\approx \frac{u_i^{n+1} - u_i^n}{\Delta t} \\
            \frac{\partial^2 u}{\partial x^2} &\approx \frac{u_{i+1}^n - 2u_i^n + u_{i-1}^n}{(\Delta x)^2}
        \end{align*}
        \vspace{0.3cm}
        \item Leads to explicit scheme:
        \[ u_i^{n+1} = u_i^n + \alpha\frac{\Delta t}{(\Delta x)^2}(u_{i+1}^n - 2u_i^n + u_{i-1}^n) \]
    \end{itemize}
\end{frame}

\begin{frame}{Numerical Implementation}
    \begin{algorithm}[H]
        \begin{algorithmic}[1]
            \State \textbf{Initialize:} Domain length $L$, grid points $N_x$, time $T$
            \State Set $\Delta x = L/N_x$, $\alpha$, $\Delta t \leq \frac{(\Delta x)^2}{2\alpha}$
            \State Create spatial grid: $x_i = i\Delta x$, $i = 0,\ldots,N_x$
            \State Set initial condition: $u_i^0 = f(x_i)$
            \State Set boundary conditions: $u_0^n = u_{N_x}^n = 0$
            \For{$n = 0$ to $N_t-1$}
                \For{$i = 1$ to $N_x-1$}
                    \State $u_i^{n+1} = u_i^n + \alpha\frac{\Delta t}{(\Delta x)^2}(u_{i+1}^n - 2u_i^n + u_{i-1}^n)$
                \EndFor
                \State Update boundary conditions
                \If{visualization step}
                    \State Plot current solution
                \EndIf
            \EndFor
        \end{algorithmic}
        \caption{Explicit Finite Difference Method for Heat Equation}
    \end{algorithm}
\end{frame}

\begin{frame}{Stability and Convergence}
    \begin{itemize}
        \item \textbf{Stability Condition:}
        \[ \Delta t \leq \frac{(\Delta x)^2}{2\alpha} \]
        \item This is known as the CFL (Courant-Friedrichs-Lewy) condition
        \vspace{0.3cm}
        \item \textbf{Key Implementation Considerations:}
        \begin{itemize}
            \item Choose $\Delta t$ to satisfy stability condition
            \item Handle boundary conditions consistently
            \item Consider mesh refinement for accuracy
            \item Monitor solution behavior for instabilities
        \end{itemize}
    \end{itemize}
\end{frame}

\end{document}