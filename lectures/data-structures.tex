\documentclass[10pt]{beamer}
\usepackage{tabulary}
\usepackage{amsfonts}
\usepackage{color}
\usepackage{amssymb}
\usepackage{amsmath}
\usepackage{amsthm}
\usepackage{epstopdf}
\usepackage{natbib}
\usepackage{setspace}
\usepackage{booktabs}
\usepackage{tikz}
\usepackage{graphicx}
\usepackage{tabulary}
\usepackage{caption}
\usepackage{listings}
\usepackage{xcolor}

\lstset{
    language=Matlab,
    basicstyle=\footnotesize\ttfamily,
    keywordstyle=\color{blue},
    commentstyle=\color{green!60!black},
    stringstyle=\color{red},
    numbers=left,
    numberstyle=\tiny,
    frame=single,
    breaklines=true
}

%\usefonttheme{professionalfonts}
%\setbeamerfont{itemize/enumerate body}{family=\sffamily, size={\fontsize{10}{10}}}
%\setbeamerfont{itemize/enumerate subbody}{family=\sffamily, size={\fontsize{10}{10}}}
\setbeamerfont{frametitle}{size=\large}

%\mode<presentation>
%\usetheme{CambridgeUS}
\usetheme{metropolis}
\usecolortheme{default} % change this to change the color scheme
\usepackage{istgame}
%\setbeamercovered{transparent}
\setbeamertemplate{navigation symbols}{}
\usepackage{caption,subcaption}
\usepackage[flushleft]{threeparttable}
\setbeamertemplate{itemize items}[circle]
\def\sym#1{\ifmmode^{#1}\else\(^{#1}\)\fi}
% Define document properties
%\title{\textsc{Social Status, Economic Development and Female Labor Force (Non) Participation}}

%\author[Short Name (U ABC)]{%
%  \texorpdfstring{%
%    \begin{columns}
%      \column{.5\linewidth}
%      \centering
%     Kaivan Munshi \\ Yale University
%      \column{.5\linewidth}
%      \centering
%      Swapnil Singh \\ Bank of Lithuania
%    \end{columns}
% }
% {Anu Alexander, Nancy Luke, Kaivan Munshi, Swapnil Singh}
%}

%\date{April 2024}

\newcommand*\widebar[1]{%
   \hbox{%
     \vbox{%
       \hrule height 0.5pt % The actual bar
       \kern0.5ex%         % Distance between bar and symbol
       \hbox{%
         \kern-0.1em%      % Shortening on the left side
         \ensuremath{#1}%
         \kern-0.1em%      % Shortening on the right side
       }%
     }%
   }%
}

% bibtex fix
\bibliographystyle{plainnat}


% theorems etc
\theoremstyle{plain}% default
\newtheorem{thm}{Theorem}
\newtheorem{lem}{Lemma}
%\newtheorem{prop}{Proposition}
\newtheorem{cor}{Corollary}
\theoremstyle{definition}
\newtheorem{defn}{Definition}
\newtheorem{conj}{Conjecture}
\newtheorem{exmp}{Example}
\theoremstyle{remark}
\newtheorem*{rem}{Remark}
%\newtheorem*{note}{Note}
\newtheorem{case}{Case}

% References to theorems etc
\newcommand{\thmref}[1]{Theorem~\ref{#1}}
\newcommand{\lemref}[1]{Lemma~\ref{#1}}
\newcommand{\propref}[1]{Proposition~\ref{#1}}
\newcommand{\corref}[1]{Corollary~\ref{#1}}
\newcommand{\defref}[1]{Definition~\ref{#1}}

\newcommand{\propnumber}{} % initialize
\newtheorem*{prop}{Proposition \propnumber}
\newenvironment{propc}[1]
{\renewcommand{\propnumber}{#1}%
\begin{prop}
		\end{prop}}


        \title{Data Structures in MATLAB}
        \subtitle{Lecture 2}
        \author{Swapnil Singh\\LB, KTU}
        
        \date{\today}
               
\begin{document}

\begin{frame}
\titlepage
\end{frame}



\begin{frame}[fragile]{Overview}
    \begin{itemize}
        \item Numerical Arrays: Foundation of MATLAB
        \item Cell Arrays: Heterogeneous Data Containers
        \item Structures: Named Fields for Organized Data
        \item Tables: Database-like Organization
        \item Advanced Data Types: Categorical \& DateTime
    \end{itemize}
\end{frame}

\begin{frame}[fragile]{Numerical Arrays}
    \begin{columns}
        \column{0.6\textwidth}
        \begin{itemize}
            \item Matrices and vectors are MATLAB's primary data type
            \item Support for multiple numeric classes
            \item Efficient memory management
        \end{itemize}
        
        \column{0.4\textwidth}
        \begin{lstlisting}
% Creating arrays
A = [1 2 3; 4 5 6];
B = zeros(3,4);
C = rand(2,2);
        \end{lstlisting}
    \end{columns}
\end{frame}

\begin{frame}[fragile]{Cell Arrays}
    \textbf{Versatile Containers for Mixed Data Types}
    \begin{lstlisting}
% Cell array creation
data = {[1 2 3], 'text', ...
        struct('field1', 10)};
        
% Accessing elements
element1 = data{1};  % Content
element2 = data(2);  % Cell
    \end{lstlisting}
    \vspace{0.5cm}
    Key Features:
    \begin{itemize}
        \item Store different types in each cell
        \item Flexible indexing methods
        \item Dynamic size adjustment
    \end{itemize}
\end{frame}

\begin{frame}[fragile]{Structures}
    \begin{columns}
        \column{0.5\textwidth}
        \textbf{Organization:}
        \begin{itemize}
            \item Named fields
            \item Hierarchical data
            \item Multiple records
        \end{itemize}
        
        \column{0.5\textwidth}
        \begin{lstlisting}
student.name = 'John';
student.grades = [85 92 78];
student.info.age = 20;
student.info.id = 'A123';
        \end{lstlisting}
    \end{columns}
\end{frame}

\begin{frame}[fragile]{Tables}
    \textbf{Database-Style Data Organization}
    \begin{lstlisting}
% Create a table
T = table([1;2;3], ...
          {'A';'B';'C'}, ...
          [10;20;30], ...
    'VariableNames', ...
    {'ID','Name','Score'});
    \end{lstlisting}
    \begin{itemize}
        \item Self-describing data structure
        \item Built-in support for missing data
        \item SQL-like operations
    \end{itemize}
\end{frame}

\begin{frame}[fragile]{Advanced Data Types}
    \textbf{Categorical Arrays}
    \begin{lstlisting}
categories = categorical(...
    {'Small','Medium','Large'});
    \end{lstlisting}
    
    \textbf{DateTime Arrays}
    \begin{lstlisting}
dates = datetime(...
    '2024-01-01','2024-12-31');
    \end{lstlisting}
    \begin{itemize}
        \item Optimized memory usage
        \item Special comparison operations
        \item Built-in analysis functions
    \end{itemize}
\end{frame}

\begin{frame}[fragile]{Performance Considerations}
    \begin{itemize}
        \item Preallocate arrays for better performance
        \item Use appropriate data type for memory efficiency
        \item Consider vectorization over loops
        \item Use sparse matrices for large, sparse data
    \end{itemize}
    \begin{lstlisting}
% Good practice
arr = zeros(1000, 1);
for i = 1:1000
    arr(i) = i^2;
end
    \end{lstlisting}
\end{frame}

\begin{frame}[fragile]{Summary}
    \begin{itemize}
        \item Arrays form the foundation of MATLAB computing
        \item Cell arrays provide flexibility for mixed data types
        \item Structures organize data with named fields
        \item Tables offer database-like functionality
        \item Advanced types support specialized applications
        \item Choose appropriate structure based on:
            \begin{itemize}
                \item Data characteristics
                \item Performance requirements
                \item Code readability
            \end{itemize}
    \end{itemize}
\end{frame}

\end{document}