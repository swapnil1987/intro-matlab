\documentclass[10pt]{beamer}
\usepackage{tabulary}
\usepackage{amsfonts}
\usepackage{color}
\usepackage{amssymb}
\usepackage{amsmath}
\usepackage{amsthm}
\usepackage{epstopdf}
\usepackage{natbib}
\usepackage{setspace}
\usepackage{booktabs}
\usepackage{tikz}
\usepackage{graphicx}
\usepackage{tabulary}
\usepackage{caption}
\usepackage{listings}
\usepackage{xcolor}

\lstset{
    language=Matlab,
    basicstyle=\footnotesize\ttfamily,
    keywordstyle=\color{blue},
    commentstyle=\color{green!60!black},
    stringstyle=\color{red},
    numbers=left,
    numberstyle=\tiny,
    frame=single,
    breaklines=true
}

%\usefonttheme{professionalfonts}
%\setbeamerfont{itemize/enumerate body}{family=\sffamily, size={\fontsize{10}{10}}}
%\setbeamerfont{itemize/enumerate subbody}{family=\sffamily, size={\fontsize{10}{10}}}
\setbeamerfont{frametitle}{size=\large}

%\mode<presentation>
%\usetheme{CambridgeUS}
\usetheme{metropolis}
\usecolortheme{default} % change this to change the color scheme
\usepackage{istgame}
%\setbeamercovered{transparent}
\setbeamertemplate{navigation symbols}{}
\usepackage{caption,subcaption}
\usepackage[flushleft]{threeparttable}
\setbeamertemplate{itemize items}[circle]
\def\sym#1{\ifmmode^{#1}\else\(^{#1}\)\fi}
% Define document properties
%\title{\textsc{Social Status, Economic Development and Female Labor Force (Non) Participation}}

%\author[Short Name (U ABC)]{%
%  \texorpdfstring{%
%    \begin{columns}
%      \column{.5\linewidth}
%      \centering
%     Kaivan Munshi \\ Yale University
%      \column{.5\linewidth}
%      \centering
%      Swapnil Singh \\ Bank of Lithuania
%    \end{columns}
% }
% {Anu Alexander, Nancy Luke, Kaivan Munshi, Swapnil Singh}
%}

%\date{April 2024}

\newcommand*\widebar[1]{%
   \hbox{%
     \vbox{%
       \hrule height 0.5pt % The actual bar
       \kern0.5ex%         % Distance between bar and symbol
       \hbox{%
         \kern-0.1em%      % Shortening on the left side
         \ensuremath{#1}%
         \kern-0.1em%      % Shortening on the right side
       }%
     }%
   }%
}

% bibtex fix
\bibliographystyle{plainnat}


% theorems etc
\theoremstyle{plain}% default
\newtheorem{thm}{Theorem}
\newtheorem{lem}{Lemma}
%\newtheorem{prop}{Proposition}
\newtheorem{cor}{Corollary}
\theoremstyle{definition}
\newtheorem{defn}{Definition}
\newtheorem{conj}{Conjecture}
\newtheorem{exmp}{Example}
\theoremstyle{remark}
\newtheorem*{rem}{Remark}
%\newtheorem*{note}{Note}
\newtheorem{case}{Case}

% References to theorems etc
\newcommand{\thmref}[1]{Theorem~\ref{#1}}
\newcommand{\lemref}[1]{Lemma~\ref{#1}}
\newcommand{\propref}[1]{Proposition~\ref{#1}}
\newcommand{\corref}[1]{Corollary~\ref{#1}}
\newcommand{\defref}[1]{Definition~\ref{#1}}

\newcommand{\propnumber}{} % initialize
\newtheorem*{prop}{Proposition \propnumber}
\newenvironment{propc}[1]
{\renewcommand{\propnumber}{#1}%
\begin{prop}
		\end{prop}}


        \title{Vectorization}
        \subtitle{Lecture 3}
        
        \author{Swapnil Singh\\LB, KTU}
        
        \date{\today}
               
\begin{document}

\begin{frame}
\titlepage
\end{frame}


\begin{frame}[fragile]{Fundamental Concepts}
  \begin{itemize}
      \item Definition: Replacing loop-based operations with array operations
      \item Core principle: Leveraging MATLAB's optimized array operations
      \item Performance implications: Reduced overhead and improved execution speed
      \item Memory considerations: Efficient utilization of computational resources
  \end{itemize}
\end{frame}

\begin{frame}[fragile]{Vectorization Principles}
  \begin{columns}
      \column{0.6\textwidth}
      \textbf{Key Advantages:}
      \begin{itemize}
          \item Elimination of explicit loops
          \item Improved code readability
          \item Enhanced execution efficiency
          \item Better memory management
      \end{itemize}
      
      \column{0.4\textwidth}
      \textbf{Primary Applications:}
      \begin{itemize}
          \item Numerical computations
          \item Data preprocessing
          \item Signal processing
          \item Statistical analysis
      \end{itemize}
  \end{columns}
\end{frame}

\begin{frame}[fragile]{Element-wise Operations}
  \textbf{Traditional Approach vs. Vectorized Solution}
  \begin{lstlisting}
% Non-vectorized
for i = 1:length(x)
  y(i) = sin(x(i))^2 + cos(x(i))^2;
end

% Vectorized
y = sin(x).^2 + cos(x).^2;
  \end{lstlisting}
  \begin{itemize}
      \item Note the use of element-wise operators (.)
      \item Simplified syntax and improved performance
      \item Automatic parallel execution capabilities
  \end{itemize}
\end{frame}

\begin{frame}[fragile]{Matrix Operations}
  \textbf{Efficient Matrix Computations}
  \begin{lstlisting}
% Computing pairwise distances
% Non-vectorized
for i = 1:size(X,1)
  for j = 1:size(X,1)
      D(i,j) = norm(X(i,:)-X(j,:));
  end
end

% Vectorized
D = sqrt(sum((X - permute(X,[3,2,1])).^2,2));
  \end{lstlisting}
  \begin{itemize}
      \item Exploitation of MATLAB's matrix operations
      \item Dimensional analysis for optimal implementation
  \end{itemize}
\end{frame}

\begin{frame}[fragile]{Logical Indexing}
  \textbf{Advanced Selection Techniques}
  \begin{lstlisting}
% Find and replace values
% Non-vectorized
for i = 1:length(x)
  if x(i) < threshold
      x(i) = 0;
  end
end

% Vectorized
x(x < threshold) = 0;
  \end{lstlisting}
  \begin{itemize}
      \item Boolean array operations
      \item Conditional vector assignments
      \item Mask-based computations
  \end{itemize}
\end{frame}

\begin{frame}[fragile]{Broadcasting and Expansion}
  \textbf{Implicit Dimension Handling}
  \begin{columns}
      \column{0.5\textwidth}
      \begin{lstlisting}
% Broadcasting example
A = rand(100,1);
B = rand(1,100);
C = A + B;  % 100x100
      \end{lstlisting}
      
      \column{0.5\textwidth}
      \textbf{Key Concepts:}
      \begin{itemize}
          \item Automatic size matching
          \item Dimension compatibility
          \item Memory efficiency
      \end{itemize}
  \end{columns}
\end{frame}

\begin{frame}[fragile]{Performance Analysis}
  \textbf{Comparative Benchmarking}
  \begin{itemize}
      \item Execution time measurements
      \begin{lstlisting}
tic; % vectorized code; toc
      \end{lstlisting}
      \item Memory profiling techniques
      \begin{lstlisting}
profile on; % code; profile viewer
      \end{lstlisting}
      \item Optimization strategies
      \begin{itemize}
          \item Preallocating arrays
          \item Avoiding temporary arrays
          \item Utilizing built-in functions
      \end{itemize}
  \end{itemize}
\end{frame}

\begin{frame}[fragile]{Common Pitfalls}
  \textbf{Optimization Challenges}
  \begin{itemize}
      \item Excessive memory allocation in vectorized operations
      \item Inappropriate use of cell arrays in numerical computations
      \item Inefficient handling of sparse matrices
      \item Suboptimal implementation of conditional operations
  \end{itemize}
  \begin{lstlisting}
% Memory-intensive operation
result = arrayfun(@heavy_function, ...
               large_array);  % Avoid
  \end{lstlisting}
\end{frame}

\begin{frame}[fragile]{Best Practices}
  \textbf{Guidelines for Efficient Vectorization}
  \begin{columns}
      \column{0.5\textwidth}
      \textbf{Do:}
      \begin{itemize}
          \item Preallocate arrays
          \item Use built-in functions
          \item Profile code performance
          \item Consider memory usage
      \end{itemize}
      
      \column{0.5\textwidth}
      \textbf{Avoid:}
          \begin{itemize}
              \item Growing arrays incrementally
              \item Unnecessary type conversions
              \item Complex nested loops
              \item Redundant computations
          \end{itemize}
  \end{columns}
\end{frame}

\begin{frame}[fragile]{Advanced Applications}
  \textbf{Real-world Implementation Scenarios}
  \begin{itemize}
      \item Image processing operations
      \item Signal analysis algorithms
      \item Statistical computations
      \item Machine learning implementations
      \item Numerical optimization routines
  \end{itemize}
  \begin{lstlisting}
% Example: Image filtering
filtered = conv2(image, kernel, 'same');
  \end{lstlisting}
\end{frame}

\begin{frame}[fragile]{Summary}
  \textbf{Key Takeaways}
  \begin{itemize}
      \item Vectorization as a fundamental optimization strategy
      \item Balance between code readability and performance
      \item Importance of proper implementation techniques
      \item Consideration of hardware limitations
      \item Role in modern scientific computing
  \end{itemize}
\end{frame}


\end{document}